% HEAD
% Pakete laden:
\documentclass[12pt]{book}													% Schriftgroesse, Dokumenttyp
\usepackage[ngerman]{babel}													% neue Deutsche Rechtschreibung und Umlaute
\usepackage{fix-cm}															% ordentlich trennen
\usepackage[top=3cm, bottom=3cm, left=2.5cm,right=2.5cm]{geometry}					% Seitenränder formatieren
\usepackage{graphicx}														% .jpg, .pdf und andere Bildformate darstellen
\usepackage{float}															% Bilder hier und nirgends sonst
\usepackage{color}															% Farbe ( \textcolor{black, white, red, green, yellow, blue}{"`text"'} )
\usepackage{ulsy}															% Blitz: \blitza, \blitzb, ..., \blitze
\usepackage{enumerate}														% Listen: (i),(ii),... oder (a),(b),... usw. \begin{enumerate}[{(i)}]
\usepackage{amstext,amsfonts,amsopn,amsmath,amssymb,amsthm,mathrsfs,bbold}		% Mathematik
\DeclareMathAlphabet{\mathpzc}{OT1}{pzc}{m}{it}									% Zapf-Alphabet
\usepackage[pdftex,hypertexnames=false,pdfpagemode=UseThumbs,					% UseOutlines, UseThumbs, FullScreen, UseNone
linkbordercolor={1 1 1}, citebordercolor={1 1 1},pdfstartview=FitH]{hyperref}				% Fit, FitB, FitH
% Eigene Befehle:
\newcommand{\n}{\nonumber}
\newcommand{\nn}{\nonumber\\}
\newcommand{\R}{\,\Rightarrow\,}
\newcommand{\pp}[2]{\frac{\partial #1}{\partial #2}}
\newcommand{\bma}{\begin{eqnarray}}
\newcommand{\ema}{\end{eqnarray}\hspace{-0.015cm}}
\newcommand{\Def}{_{\text{def}}}
% Deutsche Anführungszeichen:
%\documentclass{article}
%\usepackage[ngerman]{babel}

\let\oldquote'
\newif\ifquoteopen
\catcode`\'=\active
\makeatletter
% we have to redefine \pr@m@s to use an active '
\def\pr@m@s{%
  \ifx'\@let@token
    \expandafter\pr@@@s
  \else
    \ifx^\@let@token
      \expandafter\expandafter\expandafter\pr@@@t
    \else
      \egroup
    \fi
  \fi}
\protected\def'{%
  \ifmmode
    \expandafter\active@math@prime
  \else
    \expandafter\active@text@prime
  \fi}
\def\active@text@prime{%
   \@ifnextchar'{%
     \ifquoteopen
       \global\quoteopenfalse\grqq\expandafter\@gobble
     \else
       \global\quoteopentrue\glqq\expandafter\@gobble
     \fi
   }{\oldquote}%
}
\makeatother


% BODY
\begin{document}
% Titelseite
\thispagestyle{empty}
\setlength{\parindent}{0pt}
\vspace{2.5cm}
{\bf \fontsize{36}{36} \selectfont {Diskrete Mathematik und }}\\[5mm]
{\bf \fontsize{36}{36} \selectfont {Logik}}\\[1cm]
{\fontsize{14}{18} \selectfont  {Universität Konstanz, Wintersemester 2022/23}}\\[1cm]
{\fontsize{18}{22} \selectfont  {Dozent: Prof.\ Dr.\ Sven Kosub}}\\[.5cm]
{\fontsize{18}{22} \selectfont  {Ausarbeitung: Dr.\ Matthias Droth}}\\[.5cm]
\newpage
% Inhaltverzeichnis
\thispagestyle{empty}
%\addtocounter{chapter}{-1}
\setcounter{secnumdepth}{2}
\setcounter{tocdepth}{3}
\numberwithin{equation}{chapter} 
\tableofcontents
% Zeilenumbruch
\setlength{\parindent}{0.0cm}			% Einrückung der ersten Zeile eines neuen Absatzes
\parskip=0.3cm						% Leerzeile im TeX-Code wird auch als Leerzeile ausgegeben
% Start %%%%%%%%%%%%%%%%%%%%%%%%%%%%%%%%%%%%%%%%%%%%%%%%%%%%%%%%%%%%%%%%%%%%%%%%%%%%%%%%%%%%%%%%%%%
\chapter{Mathematische Grundlagen}
%
\section{Zuweisung}
\begin{itemize}
\item Zuweisung als Standardform (linke Seite wird durch rechte Seite definiert):
\bma
x=\Def y\quad\text{oder}\quad X:=y\,.
\ema
$\Rightarrow$ ''$x$'' ist der Name für $y$.
\item $x$ und $y$ dürfen beliebig vertauscht werden.
\end{itemize}
Beispiele:
\begin{enumerate}
\item $x=\Def2$,
\item $x=\Def 2n+1$,
\item $f(x)=\Def 2n+1$,
\item $p|q\Leftrightarrow\Def\,\exists k:\,q=k\cdot p$ (es gibt ein $k$ mit $q=k\cdot p$).
\end{enumerate}
Beachte: ''$x=y$'' \textbf{behauptet} eine Gleichheit $\Rightarrow$ Beweis nötig!
%
\section{Iteration}
\begin{itemize}
\item Definitionsform zum Ausdrücken von Wiederholungen in Variablen, aber mit bestimmten Grenzen:
\bma
\sum_{k=1}^na_k=\Def a_1+a_2+a_3+\ldots+a_n\,,\quad\prod_{k=1}^na_k=a_1\cdot a_2\cdot\ldots\cdot a_n\,.
\ema
Beispiel: $n!=\Def\prod_{k=1}^nk\,$.
\item Typisches Problem: Finde werggleichen Ausdruck onhe die Laufvariable $k$.\\Besipiel: $\sum_{k=1}^nk=\frac{n\cdot(n+1)}{2}$. 
\end{itemize}
%
\section{Rekursion}
\begin{itemize}
\item Definitionsform, bei der die definierte Seite auf der definierenden Seite vorkommen darf: $x=\Def \ldots x\ldots$.
\item Um unendliche Schachtelungen auszuschließen werden Abbruchbedingungen festgelegt. Ein paar Beispiele:
\begin{enumerate}
\item $n!=\Def n\cdot(n-1)!$ für $n\geq1$ und $0!=\Def1$.
\item $\text{Euklid}(m,\,n)=\Def\left\{
	\begin{array}{ll}
		\text{Euklid }(\text{mod}$(n,\,m),\,m)$ & \text{falls $m\nmid n$,}\\
		m & \text{falls } m|n.
	\end{array}
\right.$
\item Fibonacci Reihe: $F_n=\Def F_{n-1}+F_{n-2}$ für $n\geq2$ und mit $F_0=\Def0,\, F_1=\Def1$.
\bma
\Rightarrow F_5&=&F_4+F_3\nn
&=&(F_3+F_2)+(F_2+F_1)\nn
&=&((F_2+F_1)+(F_1+F_0))+((F_1+F_0)+F_1)\nn
&=&(((F_1+F_0)+F_1)+(F_1+F_0))+((F_1+F_0)+F_1)\nn
&=&5F_1+3F_0=5\cdot1+3\cdot0=5\,.
\ema
\bf{Merke}: \rm{Rekursionen werden ausgewertet, indem sie in Iterationen umgewandelt werden.}
\item Ackermann Funktion (auf natürlichen Zahlen $x,\,y$):
\bma
A(0,\,y)&=\Def&y+1\,,\nn
A(x,\,0)&=\Def&x\,,\nn
A(x,\,y)&=\Def&A\left(x-1,\, A(x,\,y-1)\right)\,,\quad\text{für $x\geq1$, $y\geq1$.}
\ema
\end{enumerate}
\end{itemize}
% Finish %%%%%%%%%%%%%%%%%%%%%%%%%%%%%%%%%%%%%%%%%%%%%%%%%%%%%%%%%%%%%%%%%%%%%%%%%%%%%%%%%%%%%%%%%%


% Literatur
\newpage
\begin{thebibliography}{}
\bibitem{Baym}
G. Baym, \emph{Lectures on Quantum Mechanics} (Addison Wesley, 1993).
\bibitem{FS2}
F. Schwabl, \emph{Quantenmechanik} und \emph{Quantenmechanik f\"ur Fortgeschrittene} (Springer, 2007 und 2005).
\bibitem{Nol}
W. Nolting, \emph{Grundkurs Theoretische Physik}, B\"ande 4, 5/2 und 7 (Springer, 2005, 2006 und 2005).
\bibitem{Mesh2}
A. Messiah, \emph{Quantenmechanik 2} (de Gruyter, 1990).
\bibitem{BF}
H. Bruus and K. Flensberg, \emph{Many-Body Quantum Theory in Condensed Matter Physics: An Introduction} (Oxford, 2004).
\end{thebibliography}

\end{document}