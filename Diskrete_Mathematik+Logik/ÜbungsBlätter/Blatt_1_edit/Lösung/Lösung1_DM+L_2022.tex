\documentclass[]{article}
\usepackage{amstext,amsfonts,amsopn,amsmath,amssymb,amsthm,mathrsfs,bbold}
\usepackage{graphicx}
\usepackage{xcolor}
\usepackage{ulem}
\usepackage[top=1.5cm, bottom=2cm, left=2.2cm,right=2.2cm]{geometry}
\usepackage{url}
\parskip=5mm
\parindent=0mm
% <Eigene Befehle>
\newcommand{\n}{\nonumber}
\newcommand{\nn}{\nonumber\\}
\newcommand{\R}{\,\Rightarrow\,}
\newcommand{\pp}[2]{\frac{\partial #1}{\partial #2}}
\newcommand{\bma}{\begin{eqnarray}}
\newcommand{\ema}{\end{eqnarray}\hspace{-0.015cm}}
\newcommand{\Def}{_{\text{def}}}
% </Eigene Befehle>
\begin{document}
{\Large\bf Diskrete Mathematik und Logik -- Lösung Blatt 1}\\%[+0.2cm]
\url{https://tinyurl.com/2bx3rm4v}\\
$\text{}$\hspace{12cm}Name:\hspace{1.82cm}Matthias Droth\\
$\text{}$\hspace{12cm}Matrikelnummer: 595449\\
%%%%%%%%%%%%%%%%%%%%%%%%%%%%%%%%%%%%%%
{\bf\underline{Aufgabe 1:} Modulare Arithmetik}\\[-0.8cm]
\begin{list}{}{}
\item[a)]
Bestimmen Sie $\text{mod}(41,\,13)$.
\item[b)]
Bestimmen Sie $\text{mod}(-18,\,13)$.
\item[c)]
Bestimmen Sie die kleinste natürliche Zahl $k>0$ mit $\text{mod}(4^k,\,13)=1$.
\item[d)]
Bestimmen Sie $\text{mod}(4^{111},\,13)$.
\item[e)]
Bestimmen Sie $\text{mod}(4^{57}\cdot17^{113},\,13)$.
\end{list}
%%%%%%%%%%%%%%%%%%%%%%%%%%%%%%%%%%%%%%
{\bf Lösung}\\[-0.8cm]
\begin{list}{}{}
\item[a)]
$\text{mod}(41,\,13)=\text{mod}(3\cdot13+2,\,13)=2$.
\item[b)]
$\text{mod}(-18,\,13)=\text{mod}(-2\cdot13+8,\,13)=8$.
\item[c)]
Die Lösung ist $k=6$, denn $4^6=4096=4095+1=(13\cdot315)+1$.
\item[d)]
Das Resultat von $\text{mod}(4^i,\,13)$ wiederholt sich nach allen 6 Inkrementen\footnote{Wird ggf.~an anderer Stelle bewiesen.}von $i$. Für $i\in[1,\,2,\,3,\,4,\,5,\,6,\,7]$ ergibt $\text{mod}(4^i,\,13)$ die Resultate (in entsprechender Reihenfolge) $i\in[4,\,3,\,12,\,9,\,10,\,1,\,4]$. Da $108$ ein Vielfaches von $6$ ist, gilt also $\text{mod}(4^{108},\,13)=1$. Mit 3 weiteren Inkrementen folgt $\text{mod}(4^{111},\,13)=12$.
\item[e)]
3
\end{list}
%%%%%%%%%%%%%%%%%%%%%%%%%%%%%%%%%%%%%%
{\bf\underline{Aufgabe 2:} Induktion}\\[+0.3cm]
Zeigen Sie mit vollständiger Induktion über $n$ folgende Aussagen:\\[-0.8cm]
\begin{list}{}{}
\item[a)]
$\sum_{k=0}^n(2k{+}1)^2=\frac{(2n+1)(2n+2)(2n+3)}{6}$ für alle natürlichen Zahlen $n$.
\item[b)]
$\sum_{k=0}^n\frac{k}{(k+1)!}=\frac{(n+1)!-1}{(n+1)!}$ für alle natürlichen Zahlen $n$.
\item[c)]
$\prod_{k=2}^n\left(1-\frac{1}{k}\right)=\frac{1}{n}$ für alle natürlichen Zahlen $n\geq2$.
\item[d)]
Für alle natürlichen Zahlen $n$ ist $n^3-n$ durch 6 teilbar.
\item[e)]
Für alle natürlichen Zahlen $n$ ist $2^{3n}+13$ durch 7 teilbar.
\end{list}
%%%%%%%%%%%%%%%%%%%%%%%%%%%%%%%%%%%%%%
{\bf Lösung}
\begin{list}{}{}
\item[a)]
Induktionsanfang ($n=0$):
\bma
n=0\quad\Rightarrow\quad\sum_{k=0}^{n=0}(2k+1)^2=\frac{(2\cdot0+1)(2\cdot0+2)(2\cdot0+3)}{6}=\frac{(2n+1)(2n+2)(2n+3)}{6}=1\,.\quad\checkmark
\ema
Induktionsschritt ($n{-}1\rightarrow n$):
\bma
\sum_{k=0}^n(2k+1)^2&=&\left(\sum_{k=0}^{n-1}(2k+1)^2\right)+(2n+1)^2\nn
&=&\frac{2n\cdot2(n+1)\cdot2(n+2)}{6}+\frac{6(2n+1)^2}{6}\nn
&=&\frac{8n^3+24n^2+22n+6}{6}\nn
&=&\frac{(2n+1)(2n+2)(2n+3)}{6}\,.\qquad\square
\ema
\item[b)]
Induktionsanfang ($n=0$):
\bma
n=0\quad\Rightarrow\sum_{k=0}^{n=0}\frac{k}{(k+1)!}=\frac{0}{1!}=0=\frac{(n+1)!-1}{(n+1)!}\,.\quad\checkmark
\ema
Induktionsschritt ($n{-}1\rightarrow n$):
\bma
\sum_{k=0}^n\frac{k}{(k+1)!}&=&\left(\sum_{k=0}^{n-1}\frac{k}{(k+1)!}\right)+\frac{n}{(n+1)!}\nn
&=&\frac{n!-1}{n!}+\frac{n}{(n+1)!}\nn
&=&\frac{(n!-1)(n+1)+n}{(n+1)!}\nn
&=&\frac{(n+1)!-1}{(n+1)!}\,.\qquad\square
\ema
\item[c)]
Induktionsanfang ($n=2$):
\bma
\prod_{k=2}^2\left(1-\frac{1}{k}\right)=1-\frac{1}{2}=\frac{1}{2}=\frac{1}{n}\,.\quad\checkmark
\ema
Induktionsschritt ($n{-}1\rightarrow n$):
\bma
\prod_{k=2}^n\left(1-\frac{1}{k}\right)&=&\left(\prod_{k=2}^{n-1}1-\frac{1}{k}\right)\left(1-\frac{1}{n}\right)\nn
&=&\frac{1}{n-1}\cdot\left(1-\frac{1}{n}\right)=\frac{1}{n-1}\cdot\frac{n-1}{n}=\frac{1}{n}\,.\qquad\square
\ema
\item[d)]
Induktionsanfang ($n=0$):
\bma
\frac{n^3-n}{6}=\frac{0}{6}=0\in\mathbb{N}_0\,.\quad\checkmark
\ema
Induktionsschritt ($n{-}1\rightarrow n$):
\bma
\frac{\left((n-1)+1\right)^3-\left((n-1)+1\right)}{6}=\frac{n^3-n}{6}\,.\qquad\square
\ema
\item[e)]
Induktionsanfang ($n=0$):
\bma
\frac{2^{3n}+13}{7}=\frac{2^0+13}{7}=\frac{14}{7}=2\in\mathbb{N}_0\,.\quad\checkmark
\ema
Induktionsschritt ($n{-}1\rightarrow n$):
\bma
\frac{2^{3n}+13}{7}&=&\frac{2^3\cdot2^{3(n-1)}+13}{7}\nn
&=&\frac{(7+1)\cdot2^{3(n-1)}+13}{7}\nn
&=&\underbrace{7\cdot\frac{2^{3(n-1)}}{7}}_{=2^{3(n-1)}\in\mathbb{N}_0}+\underbrace{\frac{2^{3(n-1)}+13}{7}}_{\in\mathbb{N}_0}\in\mathbb{N}_0\,.\qquad\square
\ema
\end{list}
%%%%%%%%%%%%%%%%%%%%%%%%%%%%%%%%%%%%%%
{\bf\underline{Aufgabe 3:} Ackermann-Funktion}\\[+0.3cm]
Die Ackermann-Funktion $A(x,\,y)$ sei für natürliche Zahlen $x,\,y$ wie folgt rekursiv definiert:
\bma
A(0,\,y)&=\Def&y+1\nn
A(x,\,0)&=\Def&x\quad\text{falls $x\geq1$}\nn
A(x,\,y)&=\Def&A(x-1,\,A(x,\,y-1))\quad\text{falls $x\geq1$, $y\geq1$}
\ema
Zeigen Sie mittels vollständiger Induktion über $n$, dass für alle natürlichen Zahlen $n,\,y$ die folgenden Aussagen gelten:
\begin{list}{}{}
\item[a)]
$A(2,\,n)=n+2$.
\item[b)]
$A(3,\,n)=2n+3$.
\item[c)]
$A(4,\,n)=7\cdot2^n-3$.
\item[d)]
$y<A(n,\,y)$.
\item[e)]
$A(n,\,y)<A(n+1,\,y)$.
\end{list}
\textit{Hinweis}: Benutzen Sie für jede Teilaufgabe die in der vorangegangenen Teilaufgabe bewiesene Aussage. Es gilt $A(1,\,n)=n+1$ (siehe Skriptum \textit{Diskrete Mathematik und Logik}).\\[+0.5cm]
%%%%%%%%%%%%%%%%%%%%%%%%%%%%%%%%%%%%%%
{\bf Lösung}
\begin{list}{}{}
\item[a)]
Induktionsanfang ($n=0$):
\bma
A(2,\,n)=A(2,\,0)\stackrel{\stackrel{2{\geq}1}{\downarrow}}{=}2=n+2\,.\quad\checkmark
\ema
Induktionsschritt ($n{-}1\rightarrow n$):
\bma
A(2,\,n)&=&A(1,\,\underbrace{A(2,\,n-1)}_{=(n-1)+2=n+1}\nn
&=&A(1,\,n+1)\nn
&=&A(0,\,A(1,\,n))\nn
&=&A(1,\,n)+1\nn
&=&n+1+1\nn
&=&n+2\,.\qquad\square
\ema
\item[b)]
Induktionsanfang ($n=0$):
\bma
A(3,\,n)=A(3,\,0)=3=3+2\cdot n\,.\quad\checkmark
\ema
Induktionsschritt ($n{-}1\rightarrow n$):
\bma
\ldots
\ema
\item[c)]
Induktionsanfang ($n=0$):
\bma
A(4,\,0)=4=7\cdot1-3=7\cdot2^n-3\,.\quad\checkmark
\ema
Induktionsschritt ($n{-}1\rightarrow n$):
\bma
\ldots
\ema
\item[d)]
Induktionsanfang ($n=0$):
\bma
A(n,\,y)&=&A(0,\,y)=y+1>y\nn
\Rightarrow\quad y&<&A(n,\,y)\quad\text{für $n=0$.}\quad\checkmark
\ema
Induktionsschritt ($n{-}1\rightarrow n$):
\bma
\ldots
\ema
\end{list}
%%%%%%%%%%%%%%%%%%%%%%%%%%%%%%%%%%%%%%
\end{document}


















